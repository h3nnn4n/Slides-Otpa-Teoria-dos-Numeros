\documentclass{beamer}

\usepackage{graphicx,hyperref,udesc,url}
\usepackage[utf8]{inputenc}
\usepackage[T1]{fontenc}
\usepackage{booktabs}
\usepackage[portuges]{babel}
\usepackage{algorithm,algpseudocode}
\usepackage{amsmath}

\newcommand{\Mod}[1]{\ (\text{mod}\ #1)}

\title[Teoria dos números]{Teoria dos números}

\author[Renan S. Silva]{\medskip
{\small \url{uber.renan@gmail.com}}}

\institute[UDESC]
{Departamento de Ci\^encia da Computa\c{c}\~ao \\
Centro de Ci\^encias e Tecnol\'ogias\\
Universidade do Estado de Santa Catarina}

\begin{document}

\begin{frame}
    \titlepage

\end{frame}

\section{Introdução}
\begin{frame}
    \frametitle{Teoria dos Números}

    \begin{itemize}
        \item Antigamente era chamada de Aritmética;
        \item Iniciou com Euclides em 300 B.C;
        \item Estuda os números inteiros;
        \item Atualmente possui aplicações diretas em criptografia;
            \begin{itemize}
                \item Diffie-Hellman;
                \item RSA; \item Cusvas Ellipticas; \end{itemize}
    \end{itemize}
\end{frame}

\section{Números Primos}
\begin{frame}
    \frametitle{Números Primos}

    \begin{itemize}
        \item Um número natural $n > 1$ é dito primo se ele é divisível apenas por $1$ e $n$;
        \item Pode-se dizer ainda que se $p$ é primo, para $p=a * b, a\text{ e }b \in \mathbb{N}, a < b$, segue que $a=1$ e $b = p$;
        \item $1$ não é primo;
        \item $0$ não é primo;
        \item Existem infinitos números primos;
        \item Teorema fundamental da aritimética;
    \end{itemize}
\end{frame}

\begin{frame}
    \frametitle{Testando números primos}

    Três abordagens:
    \begin{itemize}
        \item Teste de força bruta (ver código);
        \item Miller-Rabin;
        \item Crivo (ver outro código);
    \end{itemize}
\end{frame}

%\begin{frame}
    %\frametitle{Fatores primos}

    %\begin{center}
        %\Huge Código tosco;
    %\end{center}
%\end{frame}

\section{Divisibilidade}
\begin{frame}
    \frametitle{Divisibilidade}

    \begin{itemize}
        \item $a$ divide $b$, denotado por $a \textbar b$, se $\exists b \in \mathbb{N}$ tal que $a = kb$;
        \item Segue que qualquer número possui $1$ como seu menor divisor;
        \item Baseado na fatoração prima de pode-se construir a lista de todos os divisores de um número;
        \item Todo número $n \in \mathbb{N}$ pode ser escrito de forma única como $n = p_1^{a_1} *p_2^{a_2} * p_3^{a_3} * \ldots * p_m^{a_m}$, onde $p_i$ é o $i$-ésimo primo;
        \item O total de fatores de um número é dado por $\prod_{i = 1}^m (a_i + 1)$;
        \item Exemplo;
    \end{itemize}
\end{frame}

\begin{frame}
    \frametitle{Maior divisor comum, MDC ou GCD}

    \begin{itemize}
        \item O GDC de $x$ e $y$ é o maior natural $k$ tal que $x=k*a$ e $y=k*b$, para algum $a,b \in \mathbb{N}$;
        \item Denotado por $gdc(x, y)$;
        \item Se $gdc(x, y) = 1$, então $x$ e $y$ são primos relativos;
        \item Se $x \textbar y$, $gdc(x, y) = x$;
        \item Se $a = bt + r$ para $t,r \in \mathbb{N}$, então $gcd(a, b) = gcd(b, r)$;
        \item Com base nas observações acima, Eclides apresentou o que é aceito por muitos como o primeiro algoritmo da história: O algoritmo de euclides;
        \item Código;
    \end{itemize}
\end{frame}

\begin{frame}
    \frametitle{Mínimo multiplo comum, MMC ou LCM}

    \begin{itemize}
        \item Útil para detectar periodicidade simultanena de dois eventos periodiocos;
        \item Denotado por $lcm(x, y)$;
        \item Segue que $lcm(x, y) \geq max(x, y)$;
        \item Sabe-se que $x*y$ é um múltiplo de $x$ e $y$, logo $lcm(x, y) \leq x*y$;
        \item Tem-se que $lcm(x, y) = \frac{x*y}{gcd(x, y)}$;
        \item Dijkstra tem um algoritmo que não utiliza multiplicação, evitando assim um possível \textit{overflow};
    \end{itemize}
\end{frame}

\section{Aritmética modular}
\begin{frame}
    \frametitle{Aritmética modular}

    \begin{itemize}
        \item $(x + y) \mod n = ((x \mod n) + (y \mod n) \mod n)$;
        \item $(x - y) \mod n = ((x \mod n) - (y \mod n) \mod n)$;
        \item $(x * y) \mod n = ((x \mod n) * (y \mod n) \mod n)$;
        \item Divisão é treta;
        \item $x^y \mod n = (x \mod n)^y \mod n$;
        \item Pode ser usado para determinar os ultimos digitos de um número;
        \item RSA e Diffie-Hellman;
    \end{itemize}
\end{frame}

\begin{frame}
    \frametitle{Congruencias}

    \begin{itemize}
        \item É uma forma de representar a aritmética modular;
        \item Por definição $a \equiv b \Mod{n}$ se $n \textbar (a - b)$;
        \item Se $a \mod n = b$ então $a \equiv b \Mod{n}$;
        \item Se $a \equiv b \Mod{n}$ e $c \equiv d \Mod{n}$ então $(a + c) \equiv (b + d) \Mod{n}$;
        \item Adição é uma adição com números negativos;
        \item Se $a \equiv b \Mod{n}$ então $a*d \equiv b*d \Mod{n}$
        \item Se $a \equiv b \Mod{n}$ e $c \equiv d \Mod{n}$ então $(a * c) \equiv (b * d) \Mod{n}$;
        \item Quando que $2*x \equiv 3 \Mod{9}$? E $2*x \equiv 3 \Mod{4}$?
    \end{itemize}
\end{frame}

\section{Equações Diofantinas}
\begin{frame}
    \frametitle{Equações Diofantinas}

    \begin{itemize}
        \item São equações com o dominio limitado aos inteiros;
        \item $a^n + b^n = c^n$;
        \item Divisão é um problema;
        \item Décimo problema de Hilbert;
        \item Algumas possuem solução polinomial:
            \begin{itemize}
                \item $ax - ny = b$;
                \item $x^2 - ny^2 = \pm 1$;
            \end{itemize}
    \end{itemize}
\end{frame}

\section{Problemas}
\begin{frame}
    \frametitle{Problemas do URI}

    \begin{minipage}{.4\textwidth}
        URI
        \begin{itemize}
            \item 1381;
            \item 1807;
            \item 1221;
            \item 1760;
            \item 1308;
        \end{itemize}
    \end{minipage}
    \begin{minipage}{.4\textwidth}
        Project Euler
        \begin{itemize}
            \item 66 Equação Diofantina;
            \item 108 Equação Diofantina;
            \item 110 Equações Diofantina;
            \item 142 Relações entre números;
            \item 97; Powermod
        \end{itemize}
    \end{minipage}
\end{frame}

\end{document}
